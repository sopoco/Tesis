\chapter{Conclusiones y Trabajos Futuros}
\newpage
	\section{Conclusión}

	En términos específicamente de esta tesis se logró cumplir en un gran grado el objetivo general, puesto que se mejoró la automatización del proceso de pago de facturas, esto fue logrado principalmente cambiando el ingreso de datos de la primera actividad del tramite en Engage por medio de la lectura de los XML generados por SII y por el OCR, que anteriormente tanto la obtención de la factura y el ingreso de estas eran realizadas manualmente. Mediante el desarrollo de este proyecto y finalmente observando la aplicación funcionando tanto localmente durante las pruebas como en los servidores de la empresa se puede decir que:
	
	\begin{itemize}
		\item En todo desarrollo una buena retroalimentación con el cliente es fundamental durante todas las etapas del desarrollo, en el caso de esta tesis el cliente era el propio holding de SoluNegocios, por lo tanto esto facilitó bastante el trabajo, ya que los stakeholders estaban muy cerca y podían responder a mis preguntas en cualquier momento, siendo de mucha ayuda en el análisis y diseño de lo que se pretendía realizar en ese momento. 
		
		\item Para poder construir cada parte del módulo de pago proveedores  fue necesario adquirir algunas competencias en ciertas herramientas usadas en Solu4B, esto dice de la importancia de tener la capacidad de adaptarse a lo que se va a crear y como se realizara  toma gran importancia en cualquier empresa en la actualidad, especialmente en el área de la informática, la cual avanza a una velocidad más que considerable.
		
		\item La gestión de los distintos documentos en la empresas tienen gran relevancia en su economía, dependiendo de como se gestione esto podría significar una reducción del esfuerzo que aplica la empresa, especialmente el área de contabilidad, y por ende ganando eficiencia tanto manejando documentos como en otras tareas de igual importancia.
		
		\item La importancia que posee el manejo de las base de datos \textit{(SQL Server)} en Solu4b para trabajar con ENGAGE es unos de los puntos mas importantes para el desarrollo de cualquier aplicación que se desarrolle con esta herramienta, como mínimo se debe tener un conocimiento básico para poder realizar cualquier tipo de consultas, crear procedimientos almacenados y vistas. Todo esto fue necesario para integrar mi aplicación que descarga las facturas electrónicas con ENGAGE y cumpliendo su principal intención, la cual era darle automatización al ingreso de los datos e imágenes a este proceso de la empresa.
		
		
		
%	   Primero que nada el trabajo realizado en esta tesis me permitió adquirir una nueva percepción de como realmente se trabaja en el desarrollo de una aplicación para el mundo empresarial en la actualidad, me permitió conocer nuevas herramientas, retomar y aplicar conocimientos que habían sido entregados en algunos de los cursos en la universidad.
	\end{itemize}
	
	
	\section{Trabajos Futuros}
	
	\begin{itemize}
		
		\item Terminar de implantar la parte del OCR al nuevo proceso de Pago de proveedores.
		
		\item Replicar esta tesis en otros procesos o proyectos de la empresa que impliquen gestionar varios documentos, mediante la integración de Engage y el OCR. Por ejemplo en la empresa se encuentra en proceso de desarrollo un sistema similar al para automatizar las solicitudes de crédito hipotecario y tasaciones de propiedades.
		
		\item Capacitar e informar sobre los nuevos cambios a las personas que usaran este tramite\textit{(área de contabilidad y gerentes de cada empresa)}.
	\end{itemize}
	
	