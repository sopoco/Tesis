\chapter{Estado del Arte}
\newpage
    Es bien sabido por todos que hoy en día la mayoría de las empresas del área TI, si es que no son todas, manejan
    todo sus operaciones a bases de procesos, esto ha llevado un notorio aumento de su eficiencia y eficacia, además la integración de herramientas las cuales disminuyen costos y aumentando las ganancias mediante una mejor gestión de sus, tanto como el manejo de clientes y los recursos internos de cada empresa \emph{(ej: herramientas ERP y CRM)}.
    \newline
    \par
    En la actualidad de muchas empresas, algunos de los servicios internos realizados por estas requieren cierta rapidez, eficiencia y eficacia, para ello es necesario que con la ayuda de ciertas herramientas para facilitación del control de estos procesos de forma más automatizada es por ello que muchas empresas no solo usan software ERP, CRM o BPM, si no que tratan de integrar los servicios que estas herramientas proveen con todos sus proyectos, así generando un solución más completa y de calidad. En el caso específico del marco de esta tesis se trabajó con la herramienta CRM de la empresa SoluNegocios con el cual buscó integrar con el OCR y otras herramientas para automatizar lo más posible el proceso de pago de proveedores mediante el reconocimiento de caracteres y la lectura de los XML generados por el Servicios De Impuestos Internos de Chile \textit{(S.I.I.)}.
    \newline
    \par
    Un ejemplo de lo nombrado es la herramienta llamada Engage, usada en SoluNegocios, que entre algunos de sus servicios se encuentran herramientas CRM y BPM, esta herramienta es usada como base de desarrollo para la gran mayoría de sus proyectos y da gran apoyo a la gestión de esta empresa, ya que permite saber con mayor facilidad que es lo que los clientes necesitan, y por lo tanto, como los productos ofrecidos a estos están posicionados con respecto al mercado y a la competencia.    
    \newline
    \par
    Dentro del mercado actual existen una gran variedad de herramientas que ayudan a mejorar varios aspectos de la gestión de empresas, agilizar procesos o específicamente a automatizar procesos de pago de facturas a proveedores alguno de ellos se describen en los puntos siguientes:
    
    \section{Customer Relationship Management (CRM)}
    CRM es una estrategia de negocios dirigida a entender y responder a las necesidades de los clientes actuales y potenciales de una empresa, esto gracias a la automatización y sincronización de los procesos de negocios, específicamente en el área de ventas, de comercialización, servicio al cliente y soporte. Para comprender mejor esta estrategia analice las funcionalidades de las siguientes herramientas que usan esta estrategia para los negocios.
    
	    \subsection{PeopleSoft}
	    Esta herramienta se centra básicamente en ofrecer una solución que permite transformar un conjunto de datos complejos en información que será útil para establecer estrategias de negocios.
	    \newline
	    \par
	    Gracias a la implementación de esta aplicación, determinados sectores claves de la empresa orientados al servicio al cliente, como son las áreas de mercadotecnia, ventas y soporte, logran acceder a la información precisa que será necesaria para gestionar la interacción con los clientes.
	    \newline
	    \par
	    Las estrategias que surgen en base a la información brindada por este sistema, serán de vital importancia para la compañía, debido a que reportarán ventajosos resultados relacionados con el incremento de los ingresos, reducciones en inversiones inadecuadas y cargas laborales, reducción de los ciclos de venta, entre otros, es decir una exactitud casi del 100$\%$ en la oferta ofrecida al consumidor. En la Figura 2.1 se puede observar lo que ofrece PeopleSoft. (ORACLE, s.f.)
	    \newline
	    
	    \begin{figure}[H]
	    	\centering
	    	\includegraphics[width=0.5\textwidth]{Imagenes/PeopleSoft}
	    	\caption{PeopleSoft CRM (ORACLE, s.f)}
	    \end{figure}
	    
	    \subsection{Microsoft Dynamics CRM}
		Microsoft Dynamics CRM es un software de gestión de relaciones con clientes desarrollado por Microsoft. El producto se concentra principalmente en ventas, marketing, y los sectores de servicios, permite reducir costos y aumentar la rentabilidad al organizar y automatizar los procesos comerciales que nutren la satisfacción y lealtad de los clientes. 
	    La administración de las relaciones con el cliente \textit{(CRM)} ofrece una visión holística de cada uno de los clientes, lo que permite que los empleados que los atienden en persona puedan tomar decisiones rápidas e informadas acerca de los esfuerzos estratégicos en las áreas de ventas, marketing y servicio al cliente como se ve en la Figura 2.2 (MICROSOFT,s.f.).
	    
	    \begin{figure}[H]
	    	\centering
	    	\includegraphics[width=0.5\textwidth]{Imagenes/MSDynamicsCRM}
	    	\caption{Microsoft Dynamics CRM (MICROSOFT, s.f)}
	    \end{figure}
	    
	    \subsection{SugarCRM}
	    Es un sistema para la administración de la relación con los clientes \textit{(CRM)}. basado en LAMP \textit{(Linux-Apache-MySQL-PHP)}. SugarCRM es una aplicación CRM muy completa para negocios de distinto tamaño. Está diseñada para facilitar la gestión de ventas, oportunidades, contactos de negocios y además permita la integración con muchas otras herramientas de gestión (SUGARCRM, s.f)  \textit{(Ver Figura 2.3)}.
	    
	    \begin{figure}[h]
	       	\centering
	       	\includegraphics[width=0.5\textwidth]{Imagenes/sugarCRM}
	       	\caption{SugarCRM (SUGARCRM, s.f.)}
	    \end{figure}
    
    \section{Optical character recognition (OCR)}
	OCR es una tecnología que intenta imitar la capacidad de las personas de reconocer texto de algún documento, específicamente los OCR son capaces de buscar y encontrar los distintos caracteres, imágenes o códigos de barra dentro de un documento para luego extraer la información y utilizarla para lo que se crea necesario. Algunas de las software que usan esta tecnología en la actualidad son:

	    \subsection{IRISCapture}
		Es una solución para la digitalización, procesamiento y el archivado de documentos, como también de  facturas electrónicas. Esta solución permite escanear imágenes de facturas para un archivado óptimo. Posee un motor OCR que le da la posibilidad de crear documentos indexados para una recuperación fácil. Esta solución extrae la información de indexación de una factura de una sola o de varias páginas, procedentes de diferentes proveedores y países, en diferentes idiomas (IRIS, 2013).
		
		\subsection{NUANCE Omnipage}
		Esta solución consiste en el uso de un motor OCR para la gestión de todo tipo  de documentos, posee una alta precisión al momento del reconocimiento de caracteres mediante a una nueva mejora en los algoritmos de tratamientos de imágenes, como por ejemplo el algoritmo para quitar manchas permitiendo una mayor nitidez al instante del reconocimiento. Posee conexión a la nube, permitiendo la descarga y el almacenamiento de documentos ya digitalizados y/o exportación en distintos formatos según sea necesario (NUANCE, 2013).
	
		\subsection{TypeReader}
		Es una aplicación de reconocimiento óptico de caracteres a nivel corporativo, cuya su mayor fortaleza yace en la rapidez de digitalización, llegando a una velocidad de 8000 páginas por hora, Se enfoca principalmente en la captura de grandes lotes documentos, dando menor importancia a la calidad de la digitalización que a la eficiencia del reconocimiento de caracteres.
		(EXPER, s.f)
	\section{Business Process Management (BPM)}
	Corresponde a la gestión de procesos de negocios utilizando métodos, técnicas y software para diseñar, ejecutar, controlar y analizar procesos operacionales que involucran personas, organizaciones, aplicaciones, documentos y otras fuentes de información. (Van der Aalst, Ter Hofstede, and Weske; 2003)
	
		\subsection{Bonita BPM}
		Bonita BPM Studio es un entorno gráfico para la creación de procesos. Contiene tres herramientas para el diseño de un workflow (BONITASOFT, s.f) :
		\begin{itemize}
			\item Bonita Studio: permite al usuario modificar gráficamente los procesos de negocio siguiendo el estándar BPMN. Éste puede conectar procesos a otras partes del sistema de información \textit{(mensajería, ERP, ECM, bases de datos,etc)} para generar una aplicación de negocios autónoma accesible como formulario web. 
			\item Bonita BPM Engine: El BPM Engine es un  API  JAVA que permite al usuario interactuar con el proceso o los procesos.
			\item Bonita User Experience: es un portal web que permite a cada usuario final gestionar en una interfaz similar a la de un correo web todas las tareas y procesos en las cuales la persona está involucrado.
		\end{itemize}
		
		\subsection{ProcessMaker}
		ProcessMaker una herramienta totalmente libre y de código abierto \textit{(Open Source)}, disponible para las pequeñas y medianas empresas que necesiten de una herramienta informática capaz de colaborar con las actividades y procesos que realizan. ProcessMaker, permite a personas sin experiencia en programación, diseñar y aplicar soluciones para los procesos que se realizan en la misma.
		\newline
		\par
		Dentro de las posibilidades de la aplicación, se puede destacar que permite una forma sencilla de administrar los flujos de trabajo y ahorrar tiempo a la empresa, enfocándose esta, en cosas mucho más importantes. Además, permite adaptar sus módulos y elementos a cualquier organización, pues posee un código y estructura de libre manejo. (COLOSA,s.f)
		
		\subsection{Oracle BPM Suite}
		Anteriormente llamado Aqualogic BPM Suite, cuenta con un set de herramientas para la administración de procesos de negocio. Combina el flujo de trabajo y tecnología de procesos con una aplicación funcional. Oracle BPM Suite es actualmente software privativo por lo que se debe pagar una licencia. Algunas de sus principales caracteristicas es que cuenta con: (ORACLE, 2012)
		
		\begin{itemize}
			\item Proceso de iniciación/Terminación de actividades.
			\item Actividades de interacción humana.
			\item Actividades de interacción con el sistema. 
			\item Actividades de interacción organizacional.
			\item Actividades de control de procesos.
			\item Actividades globales.
			\item Actividades misceláneas.
		\end{itemize}
	
	