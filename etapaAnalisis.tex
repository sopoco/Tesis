\section{Resolución  del Problema}
	
	\subsection{Etapa de Analisis}

	\subsubsection{Reuniones} 
	Principalmente durante de la primera etapa se realizaron reuniones para obtener lo requerimientos del sistema a realizar, luego se efectuaron una series de reuniones para aclarar dudas, corregir los modelos que se fueron construyendo a medida se avanzaba en el proyecto de tesis y para controlar el avance del trabajo.
		
	\subsubsection{Revisión y análisis del proceso actual de pago} 
	Luego de una serie de reuniones nombradas en el punto anterior se procedió a revisar y analizar los requerimientos obtenidos de estas y se concluyo a grandes rasgos que lo que se deseaba desarrollar es un un nuevo proceso mas automatizado para el pago y visación de facturas a proveedores. Se concluyó que este proceso consta de 4 etapas o actividades las cuales se muestran en la Figura 4.2 y 4.3.

	\begin{figure}[H]
		\centering
		\includegraphics[width=0.7\textwidth]{Imagenes/procesoGeneral}
		\caption{Proceso pago y visación de facturas}
	\end{figure}
	
	\begin{figure}[H]
%		\centering
		\begin{turn}{270}
			\includegraphics[width=1.5\textwidth, height=0.95\textwidth]{Imagenes/original}
		\end{turn}
		\caption{Modelo proceso pago proveedores}
	\end{figure}

	\paragraph{Análisis de Etapa de Recepción}
	 El proceso inicia con la llegada de las facturas a la recepción de la empresa. la persona que esta en recepción \textit{(generalmente una secretaria)} después de recibir los documentos los escanea para luego ser usados en la siguiente etapa.
			
	\paragraph{Análisis de Etapa de Registro}
	 La que recepciona los documentos se encarga de ingresar a Engage las facturas que llegaron, se ingresa:
				
	\begin{itemize}
		\item El proveedor o la empresa de cual proviene la factura.
		\item El monto total.
		\item La empresa que efectuó la orden de compra.
		\item La Fecha de recepción.
		\item El gerente que autorizará esta factura.
	\end{itemize}
			
	\par
	Y luego se adjunta el archivo escaneado en la etapa anterior. Finalmente se informa a contabilidad que se hizo recepción de los documentos y son enviados a los personas encargadas  de verificar ciertas condiciones necesarias para la factura siga el correcto ciclo de pago correcto.
		
	\paragraph{Análisis de Etapa de Aprobación} 
	Esta etapa se centra en realizar un par de comprobaciones con el fin de verificar la existencia de las ordenes de compra asociadas a las facturas que fueron recepcionadas, si no pasa esta comprobación la factura es devuelta a su proveedor y si pasa la verificación sigue con la  solicitud de aprobación del pago de dicho documento al gerente de la empresa que emitió a la orden de compra. Finalmente contabilidad asigna una fecha de pago y se añade una base de datos\textit{(planilla excel)}.
			
	\paragraph{Análisis de Etapa de Pago} 
	Por ultimo cuando ya la factura fue validada por los analistas y el gerente de la empresa, esta se paga, pero esta acción se ejecuta solo 2 veces al mes en ciertas fechas, las cuales son los 1 y los 16 de cada mes, si por las fechas de pago cae un fin de semana o un día feriado, las facturas son pagadas el día hábil mas cercano.
			
			
	