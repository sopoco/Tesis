{\large \textbf{Sumario}}
\newline
\par
En el trabajo desarrollado en la presente tesis tuvo como principal objetivo desarrollar un nuevo modulo de ingreso de facturas de proveedores para Engage CRM, el cual debía tener como principal característica que ingresara los datos automáticamente, este modulo es usado por todas las empresas pertenecientes al holding SoluNegocios y con el fin de dar una mayor eficiencia y eficacia al momento de realizar la tarea de pago de proveedores. La construcción del modulo se baso en una implementación de del marco de trabajo BPM y el cual consta de 4 etapas.
\newline
\par
En la actualidad un factor clave para las empresas es su relación con los clientes. la cual permite mejorar sus ingresos, la calidad de sus servicios y poseer una información acabada de sus contactos, todo esto es lo que ofrece una herramienta que utiliza la metodología enfocada en los clientes o CRM \textit{(Customer Relationship Managment)}, por lo tanto tener una herramienta como Engage que permite tener todo lo nombrado es de gran ayuda para cualquier compañía. Además de poseer las virtudes de un CRM, Engage para dar un mejor soporte al cliente posee una herramienta llamada Designer con la cual se crean todas las aplicaciones que el CRM usara, en el caso de esta tesis el modulo también se creo con la herramienta llamada Designer.
\newline
\par
Durante el tiempo que duro el proyecto se logro completar en su mayoría lo propuesto dividiendo el trabajo en 4 etapas, la primera de ellas consistió en algunas reuniones y análisis del proceso que había de pago proveedores, luego en la siguiente etapa se procedió a construir los modelos del nuevo proceso usando notación BPM. En la tercera etapa se comenzó a construir la aplicación para obtener las facturas electrónicas y el nuevo modulo de ingreso de facturas para proveedores y finalmente la ultima etapa que consistió en el paso a producción y la corrección de errores.